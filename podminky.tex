\section*{Podmínky a použité přístroje}
Teplota v místnosti byla \SI{26.1(4)}{\degreeCelsius}.

Atmosférický tlak byl \SI{983(2)}{\kPa}.

Relativní vlhkost vzduchu byla \SI{36}{\percent}.

Zkoumaná tyč byla vyrobena z mosazi. 
Její délku jsme měřili svinovacím metrem s nejmenším dílkem \SI{1}{\mm}, který považujeme za standardní odchylku měření.

Tabelovaná hodnota hustoty mosazi je podle \cite{converter} v rozmezí \num{8400}--\SI{8750}{\kg\per\m\cubed} v závislosti na jejím složení.
O složení mosazi, z které byla vyrobena  tyč, nemáme žádné informace a proto předpokládáme rovnoměrné rozložení v tomto intervalu (standardní odchylka je rovna délce intervalu dělené odmocninou z dvanácti).
Jako hustotu tyče tedy používáme $\rho =  \SI{8580(100)}{\kg\per\m\cubed}$.

Kundtova trubice měla délku přibližně \SI{74}{\cm}.

Standardní odchylku určení rezonanční frekvence odhadujeme vždy na \SI{3}{\Hz}.

Délku rezonátoru jsme měřili vestavěným pravítkem, standardní odchylku odhadujeme $\sigma_l = \SI{0.5}{\mm}$.