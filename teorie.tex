\section*{Teoretická část}
\subsection*{Kundtova trubice}
Budeme měřit rychlost zvuku v kovové tyči pomocí Kundtovy trubice.
Kundtova trubice je z jedné strany uzavřená skleněná trubice, z druhé strany do ní vložíme tyč ze zkoumaného materiálu, kterou na konci opatříme korkovým pístem.
Do trubice rovnoměrně rozprostřeme korkový prášek a tyč podélně rozkmitáme.
Pokud v trubici vzniklo stojaté vlnění, prášek vytvoří obrazec naznačený v obrázku \ref{obr::obrazectrubice}.
Pokud stojaté vlnění nevzniklo, změníme vzdálenost mezi koncem trubice a korkovým pístem a opakujeme, dokud nevznikne.
Vzdálenost mezi dvěma nejbližšími místy, kde písek nebyl rozmetán, je rovna polovině vlnové délky zvuku.

%\figure \label{obr::obrazectrubice}

Kovovou tyč o délce $l$ upevníme v jejím prostředku, pak bude vydávat zvuk o vlnové délce $\lambda_1$ rovné dvojnásobku svojí délky, platí tedy
\begin{equation}
\lambda_1=2 \cdot l  \,.
\end{equation}

Při přechodun z jednoho prostředí do druhého si zvuk zachovává svojí frekvenci
\begin{equation}
f_1= \frac{c_1}{\lambda_1}=\frac{c_2}{\lambda_2}=f_2 \,.
\end{equation}
kde $f$ je frekvence, $c$ je rychlost zvuku a dolní indexy 1 a 2 označují prostředí (tyč, vzduch resp.).
Ze známé rychlosti zvuku ve vzduchu a změřených $\lambda_1$, $\lambda_2$ můžeme snadno určit rychlost šíření ve zkoumané tyči.
Rychlost zvuku v suchém vzduchu určíme podle vzathu \cite{ZFP}
\begin{equation}
c_2 =(\num{331.82} + \num{0.61} \cdot [t]) \si{\m\per\s} \,,
\end{equation}
kde $t$ je teplota vzduchu ve stupních Celsia.

Pro tenkou tyč platí \cite{ZFP}
\begin{equation}
c_1 = \sqrt{ \frac{E}{\rho}  } \,,
\end{equation}
kde $E$ je modul pružnosti v tahu a $\rho$ je hustota tyče.
Při známé rychlosti zvuku v tyči a její hustoty můžeme vypočítat modul pružnosti 
\begin{equation}
E=c_1^2 \cdot \rho \,.
\end{equation}


\subsection*{Uzavřený rezonátor}
Uzavřený rezonátor je dutá uzavřená kovová trubice s nastavitelnou délkou.
Na jednom jejím konci je připevněn reproduktor napojený na elektronický tónový generátor, na druhém konci je mikrofon napojený na mikroampérmetr.





